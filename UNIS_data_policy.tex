\documentclass[a4paper,english, 11pt]{article}

\usepackage[a4paper,inner=2.5cm,outer=2.5cm,top=2.5cm,bottom=2.5cm,pdftex]{geometry} 
\usepackage[colorlinks = true,
            linkcolor = blue,
            urlcolor  = blue,
            citecolor = blue,
            anchorcolor = blue]{hyperref}
\usepackage{enumerate}
\hypersetup{colorlinks=true,linkcolor=blue, linktocpage}

\newcommand{\emailme}{\href{mailto:lukem@unis.no}{lukem@unis.no}}

\title{UNIS data policy}
\date{\today\\v0.1}
\author{Luke Marsden (\emailme)}

\begin{document}
\maketitle
\tableofcontents

\newpage

\section{Preamble}
\label{s:preamble}

The University Centre in Svalbard (UNIS) is the world's northernmost higher education institution. Situated in Longyearbyen, Svalbard, students and staff at UNIS have easy access to a phenomenal natural laboratory in the high arctic. They have the opportunity to collect important data that can be of interest to those who are less conveniently situated.

UNIS includes 4 scientific departments - arctic geology, arctic geophysics, arctic technology and arctic biology.

This data policy is based on the principle of open access for any interested party to data, observations and scientific reports, free of charge. However, there may be some restrictions to
the principle due to the need of confidentiality in certain cases i.e. when necessary for reasons
of privacy, protection of endangered subjects, or intellectual property rights. Limitations to open
access will be in accordance with general principles followed by the international scientific
community and funding agencies.    

The UNIS data policy is based on basic principles decided by the \href{https://www.regjeringen.no/no/dokumenter/nasjonal-strategi-for-tilgjengeliggjoring-og-deling-av-forskningsdata/id2582412/}{National strategy for publishin and sharing research data}, the \href{https://sios-svalbard.org/sites/sios-svalbard.org/files/common/SIOS_Data_Policy.pdf}{SIOS data policy}, which in turn relates to Global Earth Observations System of Systems (GEOSS) Data Sharing Principles and compliance with the EU INSPIRE Directive 2007/2/EC and The EU Public Sector Information – PSI Directive 2003/98/EC, the Aarhus Convention on environmental
data, EU Directive 2003/4EC, the ICSU Declaration on Science and the Use of Scientific Knowledge, the International Arctic Social Science Association’s Guiding Principles on the
Conduct of Research, and the OECD Principles and Guidelines for Access to Research Data from Public Funding, April 2007. UNIS is committed to contributing to the SIOS (Svalbard Integrated Arctic Earth Observing System) Data Management System (SDMS). Hence, the UNIS data policy refers to the SIOS data policy in many places. As the SIOS data policy evolves, UNIS is committed to following this evolution. However, the UNIS data policy includes information more specific to UNIS where necessary. These principles have been decided and agreed upon by the UNIS leadership group and \href{https://orcid.org/0000-0002-9746-544X}{Luke Marsden} who was working at UNIS as a data manager at the time of writing.  

\section{Aims}
\label{s:aims}

UNIS aims to share free and ethically open scientific data with other UNIS staff and students, as well as with external users. 

\section{Principles}
\label{s:principles}

The UNIS data sharing principles are as follows:

\begin{enumerate}[I]
\item There will be full and open exchange of data, metadata and products within UNIS.
\item All data, metadata and products will be made available through the \href{https://sios-svalbard.org/metsis/search}{SIOS data catalogue}, freely with minimum time delay.
\item All data, metadata and products will be distributed free of charge.
\item All data are published using the \href{https://creativecommons.org/licenses/by/4.0/}{Creative Commons Attribution license} which is
compatible with the \href{https://data.norge.no/nlod/en}{Norwegian License for Open Government Data (NLOD)}
\item All data are published in a self describing form (e.g. CF-NetCDF, Darwin Core Archive) where possible. Where this is not possible a detailed product manual shall be linked to the dataset.
\item Data access may be restricted when data release could compromise the confidentiality
of human subjects or cause harm to endangered species or other vulnerable subjects.
\begin{enumerate}[A]
\item Data used by Masters or Ph.D students may have a maximum embargo period of 4 years before
they are published.
\item Embargo periods for data are decided by the UNIS leadership group
following the procedure described in the UNIS Data Management Plan.
\end{enumerate}
\item For datasets covered by the previous statement, discovery metadata describing the data, but not the exact location must be provided.
\item UNIS staff and students shall acknowledge in any publication or any other derived work the contribution made by those who have collected and/or created the data.
\item All datasets shall be available to all UNIS staff and students during any embargo period.
\end{enumerate}


\section{Scope}
\label{s:scope}

This data policy applies to UNIS staff and third parties contributing to UNIS or using data supplied through UNIS.

Normally, any natural or legal person or any association of legal or natural persons, including
research users, shall have a right of access to UNIS data, subject to the
principles, conditions and exceptions defined in this data policy. Exceptions must be justified.

\section{Definitions and concepts}
\label{s:def}

UNIS adopts the definitions and concepts outlined in the \href{https://sios-svalbard.org/sites/sios-svalbard.org/files/common/SIOS_Data_Policy.pdf}{SIOS Data Policy} for Dataset, Contributing data centre, Metadata, Timely manner and Life cycle management. The definition of timely manner differs slightly from SIOS in that data may have an embargo period to ensure that Masters and Ph.D. students can properly publish. This difference is covered in the \hyperref[s:principles]{Principles section}.

\section{Implementation}
\label{s:implementation}

The UNIS data management plan will describe the implementation of this data policy.

\section{Approval and Review}
\label{s:approval}

ADD IF APPROVED

This data policy was approved by the UNIS leadership group on DATE. 

This data policy shall be reviewed WHEN AND BY WHO?

\end{document} 